\section{Parallèlisation}

\par Dans un premier temps il est necessaire de trouver les points de parallèlisation 
du programme séquentiel. Pour ce faire il est nécessaire de bien connaitre l'algorithme 
utilisé.\\
l'algorithme séquentiel consiste à stocker les corps dans un tableau de structures ou une structure 
de tableaux (selon si la constante \_BODIES\_SPLIT\_DATA\_ est définies ou pas), à parcourir 
cette structure et à calculer une à une les interactions entre deux corps, puis à les déplacer.\\

\par l'interet de la parallèlisation est de pouvoir répartir la structure de donnée sur tout les
noeuds du système afin d'alléger la mémoire utilisée par chaque noeud, et ainsi augmenter le 
nombre de corps de la simulation.\\

Une fois ceci fait il faut que malgrés la séparation des données nous réussissions a calculer
les interactions entre chaque corps. Nous pouvons décidé d'utiliser une architecture centralisé 
ou distribué, sachant que nous devons effectuer une sauvegarde à chaque pas de temps si l'option 
d'éxécution --save est présente.\\

\par L'avantage de la solution centralisée c'est qu'un processus (maître) distribue le travail aux autres
(les esclaves). Ainsi ce la permet de distribuer plus de travail aux noeud plus puissant et moins aux
plus lent ; ce qui ne nous concerne pas étant donnée que les noeud sont considérés homogènes
(même puissance).\\
L'inconvénient est que cette solution fonctionne trés bien avec un petit nombre de noeuds et qu'elle 
ne tient pas le passage à l'échelle.\\
Ce pendant pour effectuer les sauvegardes de chaque pas de temps ce sera la solution adoptée. 
Toutes les valeurs calculées seront rapatriées vers le processus 0 qui se chargera d'effectuer 
l'écriture  dans le fichier cible.\\
\par Quant à la solution distribuée elle passe trés bien à l'échelle, mais il est difficile de
la rendre adaptable dynamiquement et s'utilise préférenciellement dans une architecture homogène.
\\C'est ainsi que fonctionnera le calcule des interactions des corps. Au temps $t=0$, chaque noeud 
calcule les interactions entre les corps dont il dispose localement. Au temps $t=0+dt$ on envoie chaque
processus envoie à autre noeud les données qu'il a calculé et reçoit les données d'un corps
calculées d'un autre noeud. On réitère cette operation p-2 fois ainsi chaque processus a
calculé les interactions avec tout les corps du dystème.\\

\section{Structures}

\par Le programme séquentiel nous donne le choix de la représentation en mémoires des 
corps. Grace à la macro \_BODIES\_SPLIT\_DATA\_ selon si elle est définie ou non 
les corps sont stockés soit dans une structure de tableau, soit dans un tableau de structure.\\

\par Le fait d'utiliser un tableau de structure fera qu'a l'initisation au lieu de faire huit 
MPI\_scatter() (un pour cahque champs dans le cas de tableau de strucures) on en fera qu'un seul,
mais cela induit qu'à chaque echange de messaqge entre les processus on envera trop de données.
Deplus en envoyant un $"gros"$ message, on a plus de chance de passer les emission et reception en mode
synchrone (cf section recouvrement des communications). On pourrait penser qu'avec un tableau de structure,
comme les donnée sont contigue en mémoire on gagnera du temps par raport au defaut de page. Mais 
nous n'utilisons pas tout les champs pour effectuer les calculs.

\par L'avantage de la représentation des corps en structure de tableaux fait que des données d'un même
type seront contigue en mémoire. Typiquement tout les p\_posx seront contigues. Deplus les messages, 
même s'il y en a plus qui transitent, seront de plus petites taille et permettent de simuler des 
interactions avec un plus grand nombre de corps avant de passer en mode de communication synchrone.
Cette structure permet également de selectionné les champs de données à transmettre. Globalement 
on transmettra moins de quantité d'information.

\par Pour faciliter les accès mémoire des fonctions MPI\_Gatheret(), MPI\_Scatter(), nous avons fait le 
choix d'utiliser les structures de tableau. Cependant ce choix permet également lors des calcules
de facilité l'accès aux dnnées, notamment lors de la parallèlisation de chaque processus en processus 
léger avec l'API Open\_MP.\\

\section{Topologie}

\par Dans le paragraphe $"Parallelisation"$ nous avons développé la manière dont se déroulerait 
le programme parallèlisé. Les machines à notre disposition sont reliées par un switch 
réseau ce qui nous permet d'adopter n'importe quelle topologie pour éxécuter notre 
programme. Il ne nous reste qu'a adopter la bonne.\\
Quelles sont les topologies que nous pouvons mettre en place :
\begin{enumerate}
\item Etoile
\item Anneau
\item Thor 2D
\item Thor 3D
\item Hypercube (car nous avons 16 machines dans certaine salle)
\end{enumerate}

D'après l'algorithme parallèlisé que nous avons choisi à chaque itération nous faisons p
envoie de messages et toujours au même processus, de le sujet nous propose de mettre en
place une topologie en anneau. La solution d'une topologie en anneau se pésente donc
naturellement.\\
La topologie en étoile sera utilisé uniquement pour la répartition des données à 
l'initialisation et la récupération des calculs pour la sauvegarde.\\ 
L'utilisation des autres topologies seraient over-engeneering dans le cadre de cet
algorithme.

\section{Vérifications}

\par Pour vérifier que le programme effectue des calculs juste sur les particules on peut se baser, 
sur le principe de la troisième loi de Newton :
\begin{quote}
\textit{Tout corps A exerçant une force sur un corps B subit 
une force d'intensité égale, de même direction mais de sens opposé, exercée par le corps B}
\end{quote}

\par Cette loi implique que la somme des forces s'exerçant sur A additionnée à la somme des forces 
s'éxerçant sur B est égale à  \overrightarrow{0}.\\
Par induction sur N corps on obtient que :
\begin{quote}
  \begin{center}
    $\sum_{i,j}^{n} \overrightarrow{F_{i,j}}$ = \overrightarrow{0} avec i=0, j=1, $i\not=j$ et n=|N| 
  \end{center}
\end{quote}

\par Il en découle un invariant pour notre programme parallèle de calcul d'interactions entre N corps.\\
A un instant t, avant et apres que le calcul d'un $dt$ des interactions entre les corps ait été fais 
on a :
\begin{quote}
  \begin{center}
    $\sum \overrightarrow{F}$ = \overrightarrow{0}
  \end{center}
\end{quote}

\par Cependant la précision que l'on peut obtenir avec une machine étant relative à 
la taille en octet sur laquelle sont représenté les nombres, on a:
\begin{quote}
  \begin{center}
    $\sum \overrightarrow{F} \simeq \overrightarrow{0} \pm \epsilon$
  \end{center}
\end{quote}
Nous prendrons $\epsilon = 10^{-4}$.

\section{Optimisations}

\par Dans un programme parallèle deux facteurs de gain de rapidité rentre en compte : 
\begin{enumerate}
\item les temps de communication 
\item les temps des accès mémoire
\end{enumerate}

\subsection{communication}

\par Afin de gagner du temps on peut recouvrir les temps de communication par du calcul.\\
Ainsi pendant que l'on envoie les données concernant les corps que l'on utilise pour le calcul
courant, et que l'on reçoit les données concernant les corps du prochain calcul, on peut effectuer 
le calcul des interactions entre les corps locaux et les corps que l'on emet.\\

\par En utilisant la primitives MPI\_Isend() afin de faire un envoie asynchrone et la primitive
MPI\_Recv() pour faire une reception non bloquante. Après le calcule on utilise la primitive 
MPI\_Wait() pour s'assurer de la reception par le processus de rang \textit{rank+1} des données 
que l'on lui a envoyé.\\

\par Le problème avec les primitives d'emission et de reception de message MPI est que si le message 
à envoyer ou à recevoir dépasse la taille du tampon de reception interne à l'API MPI,
elles basculent en mode synchrone ce qui empèche de recouvrire les temps de communication
par le calcul. (A l'ARI les tampon sont de taille 64 Ko environ).\\

\par Donc si on a :
\begin{quote}
  \begin{center}
    $\frac{n}{p}\cdot \frac{t_{corps}}{MAX(t_{champ})} \rangle T_{tampon\_mpi}$    
  \end{center}
\end{quote}
avec $n$ nombre de corps du système, $p$ nombre de processus, $t_{corps}$ taille en octet d'un corps,
$MAX(t_{champ})$ maximum en octet de la taille des champs de la structure représentant un corps,
et $T_{tampon\_mpi}$ taille en octet du tampon de l'API MPI,\\
alors les communication deviennent synchrone et le recouvrement des temps de communication pas le calcul devient 
impossible.\\

\subsection{Accès mémoire}

\par Pour diminuer les accès mémoire et gagner en temps d'éxécution, nous avons décrit précédement
comment utiliser le NFS de l'ARI pour optimiser le chargement en mémoire du fichier de test
des interactions.\\

\par Nous avons également fais un choix de représentation des données en mémoire, à savoir une structure 
de tableaux.\\

\par L'idéal serait de faire en sorte que lors de l'éxécution de notre programme les données et le code 
soient chargés en mémoire et y tiennent intégralement afin de limiter les accès disque.
Il s'ensuit que si l'on effectue des sauvegardes à chaque $dt$ de l'état des particules, les 
accès disque générés par celles-ci ralentissent considérablement l'éxécution du programme.\\

