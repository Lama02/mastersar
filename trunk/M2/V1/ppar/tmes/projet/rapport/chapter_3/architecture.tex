\section{Parallèlisation}
\par Dans un premier temps il est necessaire de trouver les points de parallèlisation 
du programme séquentiel. Pour ce faire il est nécessaire de bien connaitre l'algorithme 
utilisé.\\
l'algorithme séquentiel consiste à stocker les corps dans un tableau de structures ou une structure 
de tableaux (selon si la constante \_BODIES\_SPLIT\_DATA\_ est définies ou pas), à parcourir 
cette structure et à calculer une à une les interactions entre deux corps, puis à les déplacer.\\
\par l'interet de la parallèlisation est de pouvoir répartir la structure de donnée sur tout les
noeuds du système afin d'alléger la mémoire utilisée par chaque noeud, et ainsi augmenter le 
nombre de corps de la simulation.\\
Une fois ceci fait il faut que malgrés la séparation des données nous réussissions a calculer
les interactions entre chaque corps. Nous pouvons décidé d'utiliser une architecture centralisé 
ou distribué, sachant que nous devons effectuer une sauvegarde à chaque pas de temps si l'option 
d'éxécution --save est présente.\\
\par L'avantage de la solution centralisée c'est qu'un processus (maître) distribue le travail aux autres
(les esclaves). Ainsi ce la permet de distribuer plus de travail aux noeud plus puissant et moins aux
plus lent ; ce qui ne nous concerne pas étant donnée que les noeud sont considérés homogènes
(même puissance).\\
L'inconvénient est que cette solution fonctionne trés bien avec un petit nombre de noeuds et qu'elle 
ne tient pas le passage à l'échelle.\\
Ce pendant pour effectuer les sauvegardes de chaque pas de temps ce sera la solution adoptée. 
Toutes les valeurs calculées seront rapatriées vers le processus 0 qui se chargera d'effectuer 
l'écriture  dans le fichier cible.\\
\par Quant à la solution distribuée elle passe trés bien à l'échelle, mais il est difficile de
la rendre adaptable dynamiquement et s'utilise préférenciellement dans une architecture homogène.
\\C'est ainsi que fonctionnera le calcule des interactions des corps. Au temps t=0, chaque noeud 
calcule les interactions entre les corps dont il dispose localement. Au temps t=0+dt on envoie chaque
processus envoie à autre noeud les données qu'il a calculé et reçoit les données d'un corps
calculées d'un autre noeud. On réitère cette operation p-2 fois ainsi chaque processus a
calculé les interactions avec tout les corps du dystème.\\

\section{Structures}
\par Le programme séquentiel nous donne le choix de la représentation en mémoires des 
corps. Grace à la macro \_BODIES\_SPLIT\_DATA\_ selon si elle est définie ou non 
les corps sont stockés soit dans une structure de tableau, soit dans un tableau de structure.\\
En mémoire un tableau de structure est représenté sous forme d'un tableau de pointeur référençant 
des structures, ce qui implique que les données ne sont pas contigues en mémoire. Alors qu'avec une 
structure de tableau, ces derniers leur données sont contigues.\\
Pour faciliter les accès mémoire des fonctions MPI\_Gatheret(), MPI\_Scatter(), nous avons fait le 
choix d'utiliser les structures de tableau. Cependant ce choix permet également lors des calcules
de facilité l'accès aux dnnées, notamment lors de la parallèlisation de chaque processus en processus 
léger avec l'API Open\_MP.\\

\section{Topologie}
\par Dans le paragraphe $"Parallèlisation$" nous avons développé la manière dont se déroulerait 
le programme parallèlisé. Les machines à notre disposition sont reliées par un switch 
réseau ce qui nous permet d'adopter n'importe quelle topologie pour éxécuter notre 
programme. Il ne nous reste qu'a adopter la bonne.\\
Quelles sont les topologies que nous pouvons mettre en place :
\begin{enumerate}
\item Etoile
\item Anneau
\item Thor 2D
\item Thor 3D
\item Hypercube (car nous avons 16 machines dans certaine salle)
\end{enumerate}
D'après l'algorithme parallèlisé que nous avons choisie à chaque itération nous faisons p
envoie de messages et toujours au même processus, de le sujet nous propose de mettre en
place une topologie en anneau. La solution d'une topologie en anneau se pésente donc
naturellement.\\

\section{Vérifications}


\section{Optimisations}
\subsection{communication}
\subsection{Accès mémoire}


