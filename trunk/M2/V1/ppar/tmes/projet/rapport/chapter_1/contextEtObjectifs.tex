\section{Contexte et objectifs}
\par Dans le cadre du module de programmation parallèle (PPAR)
du Master 2 systèmes et applications répartis (SAR) ; nous devons réaliser
la parallélisation d'un programme permettant de modéliser la dynamique des galaxies.
Ce problème d'astrophysique appelé aussi le problème des N-corps consiste à 
simuler les interactions dans un espace en 3 dimensions de N corps les uns avec
les autres.\\
\par Dans un premier temps, nous allons paralléliser le code séquentiel de manière 
à le rendre exécutable sur un petit ``cluster'' de machines de l'ARI. Pour cela nous 
utiliserons la bibliothèque MPI qui permet facilement de répartir les calculs sur 
plusieurs machines. Nous définirons la représentation en mémoire des corps, l'architecture des
communications entre chaques machines ; nous recouvrirons également les temps de 
communication par le calcul et nous rendrons les communications persistantes.\\
\par Une fois cela effectué, nous utiliserons le fait que chaque noeud (machine)
possède un processeur multi-coeur, nous mettrons en place une parellélisation hybride
multi-processus/multi-threads en utilisant l'API OpenMP.\\
\par En fin, une étude comparative des performances sera fournie,
développant et expliquant les différents résultats des performances obtenues.\\
